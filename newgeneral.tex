\documentclass[12pt]{article}
\usepackage[utf8]{inputenc}
\usepackage[russian]{babel}
\usepackage{amsmath}
\usepackage{dsfont}
\usepackage{amssymb}
\usepackage{hyperref}
\usepackage{lipsum}
\usepackage{cases}
\usepackage{ wasysym }
\usepackage[usenames]{color}
\usepackage{xcolor}
\usepackage{colortbl}
\usepackage{mathrsfs}
\righthyphenmin=2
\definecolor{linkcolor}{HTML}{799B03}
\definecolor{urlcolor}{HTML}{799B03}
\begin{document}
	\tableofcontents
	\newpage
	\part{3 семестр. Основы теории групп.}
		\section{Основные классы алгебраических систем}
		\hypertarget{indef:gruppoid}{\textbf{Группоид}}= множество+\hyperref[def:bin_oper]{бинарная операция} \\
		\hypertarget{indef:halfgroup}{\textbf{Полугруппа}}= Группоид+свойство ассоциативности ($(x*y)*z=x*(y*z)$) \\
		\hypertarget{indef:monoid}{\textbf{Моноид}} = Полугруппа + нейтральный элемент ($(e*x=x*e=x) ; (0+x=x+0=x)$) \\
		\section{Группа}
		\hypertarget{indef:group}{\textbf{Группа}} = Моноид + существование обратного элемента (противоположный) \\
		\hypertarget{inpro:group}{\textbf{Свойства групп:}}\begin{enumerate}
			\item Единственность нейтрального элемента
			\item Единственность обратного элемента
			\item $(x*y)^{-1}=y^{-1}*x^{-1}$
			\item $(x^{-1})^n=(x^n)^{-1} , n \in \mathds{N}$
			\item $x^n*x^m=x^{n+m} n,m \in \mathds{Z} $ 
			\item Уравнение $ax=b$ имеет единственное решение $x=a^{-1}b$ 
			$(xa=b\Rightarrow x=ba^{-1})$
			\item $xy=e \Rightarrow y=x^{-1} $\\
		\end{enumerate}
		Коммутативная группа = абелева группа (группа, в которой выполняется свойство коммутативности $\forall x, y \in \mathds{G} \Rightarrow xy=yx$)\\
		\hypertarget{indef:por_group}{\textbf{Порядок группы}} ($|\mathds{G}|$) -число элементов в группе\\
		\hypertarget{indef:por_el}{\textbf{Порядок элемента}} ($|x|)= min \{ n\in\mathds{N} :x^n=e\}$ (т.е. минимальная натуральная степень, в которую нужно возвести элемент, что бы он превратился в "единицу") \\
		
		\newpage
		\section{Подгруппа}
		
		$\mathds{H} \le \mathds{G}$;
		$\mathds{G}:\mathds{H}$ является группой относительно той же операции. (Подмножество H группы G называется \hypertarget{indef_subgroup}{\textbf{подгруппой}} этой группы, если оно само является группой относительно той же операции)
		
		Тривиальные подгруппы- это $e$ и $\mathds{G}$
		
		Если $\mathds{H}\le \mathds{G}$ и $\mathds{H} \not= \mathds{G}$, то будем писать $\mathds{H}<\mathds{G}$.
		
		\hypertarget{th:subgrop_e}{\textbf{T. о равенстве единичных элементов в группе и подгруппе}} $H<G \Rightarrow e_H=e_G$
		
		\textbf{Док-во:} $h \in \mathds{H} \rightarrow he_H=h ; he_G=h \Rightarrow  e_Gg^{-1}he_H=h^{-1}h=e_Ge_H \Rightarrow e_H=e_G $
		
		\hypertarget{th:subgroup_-1}{\textbf{Т. о равенстве обратных элементов}} $H<G, h \in H \Rightarrow h_H^{-1}=h_G^{-1}$
		
		\hypertarget{th_subgroup_cri}{\textbf{Критерий подгруппы:}} $ \daleth  H<G$ и $H<G \Leftrightarrow $ 
		\begin{equation*}
		\begin{cases}
		H \not= \varnothing 
		\\
		x \in H
		\\
		y \in H
		\end{cases}
		\end{equation*} , то $xy=H$ 
		
		$\daleth G$ -конечная группа 
		
		Пусть $H<G  \Rightarrow$ 
		\begin{equation*}
		\begin{cases}
		H \not= \varnothing  
		\\
		x,y \in H, xy \in H
		\end{cases}
		\end{equation*}
		
		$x\in H \Rightarrow x^k <H, \forall k \in N$, т.к. $|G|<\infty\Rightarrow E l,m : x^l=x^m\Rightarrow x^{l-m}\Rightarrow =e \Rightarrow e \in H$
		
		
		Если $G$- коммутативная группа, то $xH=Hx$, то $G/H=H\textbackslash G$. Можем ввести операцию в $G/H \qquad (xH)(yH)=(xy)H$
		
		
		\newpage
		\section{Отношения эквивалентности и смежные классы и все-все-все}
		
		Отношения на множестве $M$:
		$T \le M*M=\{(a,b):a,b\in M\}$
		
		$aTb$-если пара$(a,b)\in T$ 
		
		Примеры:
		\begin{enumerate}
			\item $T=\varnothing$
			\item $T=M*M$
			\item $M=R, aTb \Leftrightarrow a<=b$
			\item $M=R, aTb \Leftrightarrow b=a^2$
		\end{enumerate}
		
		T называется \hypertarget{indef:equiv}{\textbf{отношением эквивалентности}}, если оно удовлетворяет следующим условиям:
		\begin{enumerate}
			\item $aTa$ (рефлексивность)
			\item $aTb \rightarrow bTa$ (симметричность)
			\item $aTb, bTc \rightarrow aTc$ (транзитивность)
		\end{enumerate}
		
		Будем иметь ввиду вместо $aTb=a\sim b$, $T_a=\{b\in M:a\sim b\}$
		
		Теорема:
		\begin{enumerate}
			\item $a\in T_a$
			\item $\bigcup_{a\in M} T_a=M$
			\item $T_a \bigcap T_b \not= \varnothing \rightarrow T_a=T_b$
		\end{enumerate}
		
		
		Итак: M разбито на непересекающиеся подмножества $M\rightarrow M/\sim$ (факторизация)
		
		Пример: $H<G x\sim y \leftrightarrow x^{-1}y \in H \leftrightarrow \exists h\in H: x^{-1}y=h\leftrightarrow y=xh, h\in H$; таким образом $T_x=\{y:x\sim y\}=\{y:y=xh\}=xH;$
		
		Определение: $xH=\{xh, h\in H\}$
		
		$T_x=xH \rightarrow$ \hypertarget{indef:left_class}{\textbf{левым смежным классом}} по подгруппе $H$
		
		Аналогично, если $ x\sim y$ ввести по формуле $yx^{-1}\in H\rightarrow T_x=Hx$-\hypertarget{indef:right_class}{\textbf{правый смежный класс}}.
		
		
		
		
	
		
		$T_a=\{b:a\sim b\} $
		
		% %Тут что-то было..
		
		Если $(Z,+)$ и $(nZ,+)$, то $a\rightarrow a+nZ$
		
		Вместо $a\rightarrow a+nZ$ будем писать $[a]_n$ или $ \bar{a_n}$ или $\bar{a}$ или(like a pro) $a$
		
		$Z/_{\sim}=Z/_{nZ}=Z_n$
		
		Кольцо$(A,*,+)$
		$\bigoplus$-коммутативность
		$\bigodot$-дистрибутивность
		
		Бывают кольца коммутативные$(ab=ba)$ и с единицей$(ea=ae=a)$
		
		Подкольцо-это кольцо относительно тех же операций
		
		Поле-коммутативное ассоциативное кольцо с единицей $e\not= 0$.Кроме того, $\forall x\not=0 \exists x^{-1} (xx^{-1}=x^{-1}x=e)$
		
		$[a]_n+[b]_n=[a+b]_n$ и $[a]_n [b]_n=[ab]_n$
		
		\textbf{Пара необходимых теорем о отношениях эквивалентности:}
		\begin{enumerate}
					
		
		\item $	\left
			.\begin{aligned}
			a\sim a^\prime \\
			b\sim b^\prime
			\end{aligned}
			\right \} $
			$\rightarrow$
			$\begin{cases}
			[a^\prime +b^\prime]=[a+b] \\
			[a^\prime b^\prime]=[ab]
			\end{cases}$\\
		$\LHD a^\prime =a+nl; b^\prime=b+nk \\
		a^\prime+b^\prime=a+b+n(l+k)\rightarrow [a^\prime+b^\prime]=[a+b] \\
		a^\prime b^\prime=ab+n(ak+bl+nlk)\rightarrow [a^\prime b^\prime]=[ab]
		\RHD$
		
		\item $H<G,\cdot$
		
		
			$|T_x|=|T_z|=|H|$
		
		
		$\LHD xh_1=xh_2 \rightarrow h_1=h_2 \RHD$
		\end{enumerate}
		
		$\begin{aligned}
		H<G \quad x\sim y & \quad x^{-1}y\in H \leftrightarrow y\in xH \text{(\textbf{Левый смежный класс})} \\
		& \quad yx^{-1}\in H \leftrightarrow y\in Hx \text{(\textbf{Правый смежный класс})}
		\end{aligned} $
				
		Если $G$-коммутативна, то $xH=Hx$
				
		Множество левых смежных классов обозначается $G\diagup H $
				
		Множество правых смежных классов обозначается $H\diagdown G$ 
		
		$|G\diagup H|=|H\diagdown G|=$индекс подгруппы
		
		\hypertarget{th:t_lang}{\textbf{Теорема Лагранжа}} 
				Если $|G|=n<\infty \rightarrow |G|\vdots |H| \text{, что равносильно определению } |G|=|H|\cdot |_G\diagup^H| 
				\text{где группа }G\text{-конечная группа}	 $ 
			
			
		\hypertarget{sl:t_lang}{\textbf{Следствия из теоремы Лагранжа:}}\begin{enumerate} 
			\item$|G|\vdots |x|$
	
			$\LHD x\Rightarrow H=<x>, |H|=|x|
			\RHD$
			
			
			
			\item	$|H| \, \Big{|}\, |G|$
			
			
			\item	$x \in G \rightarrow |x| \, \Big| \, |G| $
			
			
			\item	$|G|=p -\text{простое число}\rightarrow G \text{циклическая группа, причем если} g\not= e \rightarrow G=<g>$
			
			
			\item	$|G|=n\\
				g\in G\rightarrow g^n=e$
			
			\item[Малая теорема Ферма 6.]
				$a^p \equiv a(mod\, p)$
			
			\item[Функция Эйлера 7.]
				Функция Эйлера $(\phi (n))$-функция, равная количеству натуральных чисел, меньших и взаимно простых с ним.
				
			\item[Т. Вильсона 8.]
				$(p-1)!+1\,\vdots \,p \leftrightarrow p-\text{простое}$
		\end{enumerate}	
						
			Пусть $H<G \text{, где }G - \forall$
			
			Пытаемся ввести операцию $(xH)(yH)=(xy)H$. Когда она корректна?\\
			Когда $\begin{cases}
			x \sim x' \\
			y \sim y'
			\end{cases} \rightarrow xy \sim x'y'$
			$xy\sim x'y'=xh_1 yh_2 \\
			xy=xh_1 yh_2 h_3 \qquad \Big{|} \cdot x^{-1}\\
			y=h_1yh_4 \\
			e=y^{-1}h_1 y h_4\\
			y^{-1}h_1y=h_5 \rightarrow \boxed{ y^{-1}Hy \le H \quad \forall y \in G \textcolor{red}{(1)} }\\
			\text{Из \textcolor{red}{(1)}} \rightarrow H \le yHy^{1} \quad \forall y \rightarrow H \le y^{-1}Hy \leftrightarrow \boxed{ y^{-1}Hy=H \quad \forall y\in G \textcolor{red}{(2)}  } \leftrightarrow \boxed{ Hy=yH \quad \forall y\in G \textcolor{red}{(3)} }$
			
			
			$H<G$ называется \hypertarget{indef:norm_gr}{\textbf{нормальной}}, если выполнено любое из \textcolor{red}{3} равносильных условий. \\
			В этом случае пишут $H\lhd G $\\		
			$\rceil H\lhd \rightarrow G/H$-группа относительно $(xH)(yH)=xyH$\\
			Группа $G/H$ называется \hypertarget{indef:fact_gr}{\textbf{фактор-группой}} группы $G$ по нормальной подгруппе $H$.
			
		
		
			\hypertarget{indef:morfizm}{\textbf{Гомоморфизм}} $\phi:G_1 \rightarrow G_2$ -если $\phi(xy)=\phi(x)\phi(y)$
			
			\textbf{Мономорфизм} = инъективный гоморфизм
			
			\textbf{Эпиморфизм} = сюръективный гомомрфизм.
			
			\textbf{Эндоморфизм} - если $G_1=G_2$
			
			\textbf{Автоморфизм} - изоморфизм+эндоморфизм
			
			$Ker\phi =\{x\in G_1 :\phi (x)=e_2\} \qquad (=\phi ^{-1}(e_2))$ -Ядро гомоморфизма
			
			$Im\phi =\{z\in G_2 ; \exists x\in G_1:\phi (x)=z\}=\{\phi (x), x\in G_1\}=\phi (G_1)$-Образ гоморфизма
			
			\hypertarget{inpro:gom}{\textbf{Свойства гомоморфизма:}}
				\begin{enumerate}
					\item $\phi (e_1)=e_2 \qquad \text{или } e_G \rightarrow e_H$
					\item $\phi (x^{-1})=(\phi (x))^{-1} \qquad \text{или } x\rightarrow y  \text{ то } x^{-1}\rightarrow y^{-1}$
					\item $| \phi (x)|\, \big{|} \, |x|$
					\item $Im\phi < H \qquad Im\phi=\{ \phi (x), x\in G \}$
					\item $Ker\phi <G  \qquad  Ker \phi=\{ \phi^{-1}(e_H) \}$
					\item $Ker\phi \lhd G$
					\item $\phi (x_1)=\phi (x_2) \leftrightarrow x_1 \equiv x_2 (mod \, Ker\phi) $
					\item $\phi- \text{мономорфизм}\leftrightarrow Ker \phi =\{e\} $
					\item $\phi :G \rightarrow H, \psi :H\rightarrow K -\text{гоморфизм} \rightarrow \psi \cdot \phi : G \rightarrow K - \text{гоморфизм}$
					\item $\phi :G\rightarrow H-\text{изомрфизм} \rightarrow \phi^{-1}-\text{изомрфизм} $
					\item $\phi \text{-изоморфизм, то } |\phi (x)|=|x|$
				\end{enumerate}
				
			Док-во 6-го св-ва:$\rceil x_1, x_2 \in Ker\phi \rightarrow x_1 x_2 \in Ker\phi \\
			\phi (x_1)=e \\
			\phi (x_2)=e \\
			\phi (x_1x_2)=\phi (x_1) \phi (x_2)=e\cdot e=e$
			
		
		\newtheorem{Th}{Теорема}
		\newtheorem{Def}{Определение}
		

		
		
		
		% % % % % % % % % % % %
		
		
		
		
		\newpage
		\section{Циклические группы}
				\hypertarget{indef:circle_group}{\textbf{Циклическая группа}} $\mathds{G}$ -если $ \exists a \in \mathds{G} : \mathds{G}=\{a^k, k \in \mathds{Z} \}$ (можно так же сказать, что циклическая подгруппа состоит из всех степеней элемента) \\
				Циклическая группа называется \hypertarget{indef:inf_circle_group}{\textbf{конечной}}, если $|\mathds{G}|<\infty; |\mathds{G}|=n \Rightarrow \mathds{G}=\{e,a,a^2, \dots , a^{n-1} \}, a^n=e$\\
				Циклическая группа называется \hypertarget{indef:noninf_circle_gtoup}{\textbf{бесконечной}}, если $|\mathds{G}|= \infty \Rightarrow \mathds{G}=\{ e,a,a^{-1},a^2,a^{-2}, \dots, a^k, a^{-k}, \dots \}$\\
				\hypertarget{f:circle}{\textbf{Необходимые формулы и утверждения о циклических группах}}:
				\begin{enumerate}
					\item $|x^k|=\frac{|x|}{(|x|,k)}$
					\item $x^n x^m=x^{n+m}; (x^n)^m=x^{nm}; x^0=e$ при $n,m \in \mathds{Z}$
					\item$G,x \quad <x>=\{x^n,\text{ где } n\in Z\}$-циклическая подгруппа группы $G$
					
				\end{enumerate} 
					
					Если $|x|=n<\infty \rightarrow <x>= \{e,x,x^2, \dots, x^{n-1} \}$
					
					Если $|x|=\infty \rightarrow <x>=\{e,x,x^{-1},x^2, x^{-2}, \dots \}$ 
					
					$<x>_n, $т.е. $|<x>_n|=|x|=n$
					
					$<x>_\infty$ т.е. $|<x>_\infty|=|x|=\infty$ См. семинар для св-в (Кострикин)
					
					$G$-циклическая группа, если $\exists x\in \mathds{G}:\mathds{G}=<x>$
					
					\hypertarget{th}{\textbf{Теоремы о циклических группах:}}
					
					\textbf{Теорема.} У циклической группы все подгруппы циклические, т.е. $G$-циклическая группа.($H<G \Rightarrow H$-циклическая группа)
					
					
					\textbf{Теорема.} $G$-циклическая группа, пусть $|\mathds{G}|=n \text{и } n\vdots k \Rightarrow \exists ! H<G:|H|=k$
		
					\begin{Th}
						$\mathds{Z}_n$-поле $\leftrightarrow n$-простое 
					\end{Th}
										
					\begin{Th}
						$\mathds{Z}_n \quad k-$обратим в $\mathds{Z}_n \leftrightarrow n$ и $k$-взаимно просты $\bigl\{ (n,k) =1 \bigr\} $
					\end{Th}
					
				$\phi :G_1 \rightarrow G_2$ называется изоморфизмом, если 
				$\begin{cases}
				\phi(gh)=\phi(g)\phi(h) \\
				\phi$-биекция$
				\end{cases} $
						
				\begin{Th}
					$\daleth G=<x>,\cdot$ \\
						-Если $|G|=\infty \Rightarrow G\cong \mathds{Z},+  \quad(\mathds{Z}=<1>)(G_1\cong G_2)$, то группы называются изоморфными.\\
						-Если $|G|=n<\infty \Rightarrow G\cong \mathds{Z}_n,+$
				\end{Th}
		
		\newpage
		\section{Подстановки}
		
			\hypertarget{el}{\textbf{Функция Эйлера}}$\bigl\{\varphi (n) \bigr\} $ равна количеству натуральных чисел, меньших чем $n$ и взаимно простых с $n$.\\
			\textbf{Теорема Эйлера} $a^{\phi(n)} \equiv 1(mod \, n) $\\
			\textbf{Свойства $\varphi(n)$:} \begin{enumerate}
				\item	$\phi(p)=p-1, p-\text{простое}$
				\item	$\phi(p^n)=p^n-p^{n-1}, p-\text{простое}$
				\item	$\phi(mn)=\phi(m)\phi(n), \qquad (m,n)=1$	
			\end{enumerate}
			$S_n$-группа подстановок(так же называют симметричной группой) \\
			$x=\{1,2,3, \dots , n\}$, $S_n$-мн-во биективных функций $\varphi :X\rightarrow X$ \\
			$\varphi= \begin{pmatrix}
			1 & 2 & 3 & 4 & \dots & n \\
			\varphi(1) & \varphi(2) & \varphi(3) & \varphi(4) & \dots & \varphi(n)
			\end{pmatrix}$
	
		
		\textbf{Примеры:} $\varphi=\begin{pmatrix}
		1 & 2 & 3 & 4 \\
		2 & 4 & 3 & 1
		\end{pmatrix}$
		
		$ \varphi(1)=2 , \varphi(2)=4, \varphi(3)=3, \varphi(4)=1$
		
		$\phi=\begin{pmatrix}
		1 & 2 & 3 & 4 \\
		4 & 2 & 1 & 3
		\end{pmatrix}$
		
		$(\varphi\phi)(1)=\varphi(\phi(1))=\varphi(4)=1$
		
		$\varphi^{-1}\varphi=\varphi\varphi^{-1}=e$
		
		$e=\begin{pmatrix}
		1 & 2 & 3 & 4 \\
		1 & 2 & 3 & 4
		\end{pmatrix}$
		
		$\varphi^{-1}=\begin{pmatrix}
		2 & 4 & 3 & 1 \\
		1 & 2 & 3 & 3
		\end{pmatrix}=\begin{pmatrix}
		1 & 2 & 3 & 3 \\
		4 & 1 & 3 & 2 
		\end{pmatrix}$
		
		$S_n$-группа  $\qquad |S_n|=n!$
		
		\textbf{Цикл}
		
		$\varphi=\begin{pmatrix}
		1 & 2 & 3 & 4 \\
		2 & 4 & 3 & 1
		\end{pmatrix}=(124)(3)=(3)(124)$
		
		$\phi=\begin{pmatrix}
		1 & 2 & 3 & 4 \\
		2 & 1 & 4 & 3
		\end{pmatrix}=(12)(34)=(34)(12)$
		
		\hypertarget{def:while}{\textbf{Независимые циклы}}-называются такие циклы циклы,у которых числа входят в один цикл, но не входят во второй цикл.\\
		Циклом длины два называется \textbf{транспозиция.}
		
		\hypertarget{th:while}{\textbf{Теоремы:}}\begin{enumerate}
		
		
			\item Независимые циклы коммутируют друг с другом (или $\alpha ,\beta$-независимые циклы $\rightarrow \alpha\beta=\beta\alpha$ )
			
		
		
			\item Если $\alpha=(i_1,i_2, \dots ,i_k)$-цикл длины $k \rightarrow |\alpha|=k$
		
		
			\item Пусть $\varphi=\alpha_1\alpha_2\dots\alpha_n$- произведение независимых циклов.\\
			$\alpha_1=(i_1,i_2,\dots, i_k)$\\
			$\alpha_2=(j_1,j_2,\dots, j_l) \rightarrow |\varphi|=$НОК$(|\alpha_1|,|\alpha_2|,\dots,|\alpha_m|)$
		
		
		\end{enumerate}
		
		\begin{Th}
			$|G|=\text{НОК}(k_1,k_2, \dots ,k_m)$
		\end{Th}
		
		Инверсия $ij$-если $i>j$, но $i$ левее $j$
		
		Подстановка $G=\begin{pmatrix}
		l_1 & l_2 & \dots & l_n \\
		j_1 & j_2 & \dots & j_n
		\end{pmatrix}$ называется четной, если сумма инверсий в верхней и нижней строках четная. Иначе- нечетная.
		
		Знак подстановки $sgnG=(-1)^{[l_1l_2\dots l_n]+[k_1k_2\dots k_n]}$
		
		$\begin{aligned} 
		G  &\text{-четная, если}  sgnG=1\\
		&\text{-нечетная, если}  sgnG=-1
		\end{aligned}$
		
		$|\alpha|=k\rightarrow sgn\alpha=(-1)^{k-1} \quad (=(-1)^{k+1})$
		
		$\alpha=(ij)=\begin{pmatrix}
		1 & 2 & \dots & i & \dots & j & \dots & n \\
		1 & 2 & \dots & j & \dots & i & \dots & n 
		\end{pmatrix}$-нечет
		
		\begin{Th}
			$G=\alpha_1 \alpha_2 \cdot \dots \cdot \alpha_n$-произведение независимых циклов.\\
			$sgnG=(-1)^{n-m} \quad (=(-1)^{n+m})$
		\end{Th}
		
		
		
		\newpage
		\section{Прямое произведение}
		
			\begin{Def}
				\textbf{Прямое произведение групп} \\
				$G_1, G_2, \dots , G_n - \text{группы} \, (\cdot) \quad G=G_1\times G_2 \times \dots \times G_n=\{ (g_1,g_2, \dots , g_n):g_i\in G_i, i=1, \dots , n  \}$ \\
				Введем операцию $(g_1,g_2, \dots , g_n) \cdot (g_1',g_2', \dots , g_n')=(g_1 g_1', g_2, g_2', \dots, g_n g_n') $
			\end{Def}
		
		\hypertarget{def:out_comp}{\textbf{Внешнее прямое произведение:}} $G=G_1 \times G_2 \times \dots \; (\cdot) \qquad G=G_1\oplus G_2 \oplus G_3 \oplus \dots \oplus G_n \; (+)$
		
		\hypertarget{inpro:out_comp}{\textbf{Свойства внешнего прямого произведения:}}
		\begin{enumerate}
			\item G-группа
			\item $\tilde{G_i}=\{ (e_1, e_2, \dots , e_{i-1}, g_i, e_{i+1}, \dots, e_n)<G \} \qquad 
			=\{e_1 \}\times \{e_2\} \times \dots \times \{e_{i-1}\} \times G_i \times \{e_{i+1}\}\dots \{e_n\}$
			\item $\tilde{G_1} \cong G_i \quad (\phi_i :G_i \rightarrow \tilde{G_i} \; \phi_i(g_i))=(e_1, e_2, \dots, e_{i-1}, g_i, e_{i+1}, \dots e_n)- \text{Изоморфизм} $
			\item $\tilde{G_i} \lhd G$
			
			\item $\forall g \in G \, \exists \tilde{g_1}\in \tilde{G_1}, \dots , \, \tilde{g_n}\in \tilde{G_n}: g=\tilde{g_1} \tilde{g_2} \tilde{g_3} \dots \tilde{g_n} $	
			\item $\forall g\in G \exists! \tilde{g_1}\in \tilde{G_1}, \dots , \, \tilde{g_n} \in \tilde{G_n}: g=\tilde{g_1} \tilde{g_2} \tilde{g_3} \dots \tilde{g_n} $
			\item $\tilde{G_i} \cap \tilde{G_j}=\{e\} \quad (i\not= j)  $
			\item $\tilde{g_1}\in \tilde{G_i}, \tilde{g_j}\in \tilde{G_j} \rightarrow \tilde{g_i}\tilde{g_j}\rightarrow \tilde{g_j}\tilde{g_i}$  

			\item $|G|=|G_1||G_2|\dots |G_n|$
			\item $|(g_1,g_2, \dots ,g_n)|=\text{НОК}(|g_1|,|g_2|, \dots , |g_n|)  $
			\item $\rceil G_1=<g_1>_{k_1}, \dots ,G_n=<g_n>_{k_n}  \qquad \text{G-циклическая группа} \Leftrightarrow \text{у} k_1, \dots , k_n \text{нет общих делителей} \quad \lhd \text{на дом} \rhd$\\
			Пример: $C_2 \times C_3 \cong C_6 \, (\cdot) $\\
			$\mathds{Z}_2 \oplus \mathds{Z}_3 \cong \mathds{Z}_6 \, (+)$ 
		\end{enumerate} 
		\begin{Th}
			$G, \, G_1, G_2, \dots , G_n < G$
		\end{Th}
		\begin{enumerate}
			\item $(6\Rightarrow 5)  $
			\item $6 \Rightarrow 7, \text{то если} \forall g\in G \exists ! g_1 \in G_1 , \dots , g_n\in G_n: g=g_1 g_2 \dots g_n  \Rightarrow G_i \cap G_j = \{e\} \quad (i\not= j)\\
			\lhd \text{От противного. Пусть } g \in G_i \cap G_j \Rightarrow \, g=ee \dots g \dots e \dots e = ee \dots e \dots g \dots e \Rightarrow g=e \Rightarrow G_i \cap G_j=\{e\} \rhd$
			\item $\begin{aligned}4\\ 7\end{aligned}  \} \Rightarrow 8 \quad \text{т.е. если } \, G_i \lhd G_j \quad   G_j\lhd G_i, G_i \cap G_j= \{e\}  \Rightarrow g_i g_j=g_j g_i (\Leftrightarrow g_i g_j g_i^{-1} g_j^{-1}=e ) $
		\end{enumerate}
		
		\begin{Def}
			\hypertarget{def:incomp}{\textbf{Внутренне прямое произведение:}} \\
			$G, \quad G_i, \dots , G_n <G \\
			G-\text{внутренне прямое произведение этих подгрупп, если} $\\
			\begin{enumerate}
				\item $\forall g\in G \exists !g_1,\dots, g_n : g=g_1g_2 \dots g_n$ 
				\item $G_i \lhd G_i; i=1, \dots n  $
			\end{enumerate}
		\end{Def}	
		
		
		\textbf{Теорема}	G изоморфно $G_1 \times G_2 \times \dots \times G_n\\ (\text{т.е. внутреннее прямое произведние изоморфно внешнему})$
		
		$\lhd  \phi : G_1 \times \dots \times G_n \rightarrow G $ \\
		$\phi(( g_1,g_2, \dots , g_n ))=g_1 g_2 \dots g_n$ \\
		$\phi \text{-гомоморфизм ?} \phi(( g_1 \dots , g_n )( h_1, \dots, h_n )=\phi (g_1 h_1, g_2 h_2, \dots , g_n h_n) = \\g_1 h_1 g_2 h_2 \dots g_n h_n $\\
		$\phi  \phi(( g_1 \dots , g_n )( h_1, \dots, h_n )=g_1 g_2 \dots g_n h_1 h_2 \dots h_n $Сие выра\-же\-ние вы\-хо\-дит из преды\-ду\-щей строки бла\-го\-даря свойству 8
		
		$\phi \text{-эпиморфизм. Пусть} g\in G \rightarrow \exists g_1, \dots g_n: g=g_1 \dots g_n \rightarrow \phi((g_1, \dots , g_n))=g_1 \dots g_n=g$ \\
		$\phi \text{-мономорфизм} \quad Ker\phi= \{( (g_1, \dots, g_n):g_1g_2 \dots g_n=e )\} \rightarrow g_1=g_2=\dots=g_n=e $ 
		
		\begin{Th}
			$* * \begin{cases}
			\forall g!=g_1 \dots g_n \quad(1)\\
			g_i g_j=g_j g_i  \qquad(2)
			\end{cases}\Leftrightarrow * \begin{cases}
			\forall g!=g_1 \dots g_n \quad (1)\\
			G_i \lhd G \qquad (2)
			\end{cases}  \\
			\text{ где} g_i\text{-элемент i-ой группы, а} g_j \text{-элемент j-ой группы}$
		\end{Th}
		$\lhd \rightarrow (1)** \rightarrow (1)* \\
		(1)**\rightarrow G_i \cap G_j=\{e\} \qquad \rceil g\in G_i \cap G_j \rightarrow g=e \cdot e  \dots e \cdot g \cdot e  \dots e = e \dots e g e \dots e \rightarrow g=e \\
		gg_ig^{-1}=g_1`g_2` \dots g_n`g_i` (g_n`)^{-1} \dots (g_1`)^{-1}  =g_i` g_i (g_1`)^{-1} \in G_i \rightarrow G_i \lhd G \rhd$ \\
		$\lhd  \leftarrow (1)*\rightarrow (1)** \rightarrow G_i \cap G_j=\{e\} \qquad G_i \lhd G\\
		\text{Требуется доказать:} g_i g_j=g_j g_i, \text{т.е.} g_i g_j g_i^{-1} g_j^{-1}=e   $
		
		G-внутреннее прямое произведение, если выполнена (*) ($ \Leftrightarrow \text{Выполнена (**)} $)
		
		Примеры: \begin{enumerate}
			\item $G=\mathds{Z},+ \qquad \text{не раскладывается в внутренние прямые суммы} \\
			G_n= n \mathds{Z} \text{-других подгупп нет} \qquad $	На дом: продумать и записать доказательство. \\
			$nm\in n\mathds{Z} \cap m \mathds{Z} \rightarrow n \mathds{Z}\cap m \mathds{Z} \not= \{0\} $
			\item $G=\mathds{C}^*,\cdot \\
			G_1=\mathds{R}_{>0}=\{x>0, x\in R\}\\
			G_2=U=\mathds{T}=\{z:|z|=1\}=\{z=e^{i\phi}\} \\
			G=G_1 \times G_2 ? \quad \text{G-коммутативна} \\
			z\in G \rightarrow z=|z|e^{i\phi} \qquad (|z|>0 \quad e^{i\phi \in U}) \\
			\rceil z=x_1 u_1=x_2 u_2 \rightarrow ? \begin{cases}
			x_1=x_2 \\
			u_2=u_2
			\end{cases}  \qquad x_1x^{-1}_2=u^{-1}_1u_2\rightarrow \begin{cases}
			x_1 x_2^{-1}=1 \\
			u_1^{-1} u_2=1
			\end{cases} \rightarrow \begin{cases}
			x_1=x_2 \\
			u_2=u_2
			\end{cases}$
			
		\end{enumerate} 	
		\begin{Th}
			$(G_1\times G_2 )/G_1 \cong G_2 $
		\end{Th}
		
		$\lhd \quad \phi : G_1 \times G_2 \rightarrow G_2 \\
		\phi(( g_1 g_2 ))=g_2 \\
		\phi \text{-гомоморфизм ?  } \quad \phi(( g_1,g_2 )(g_1`,g_2` ))=\phi(( g_1g_1`,g_2 g_2` ))=g_2g_2` \\
		\phi(( g_1, g_2 )) \cdot \phi(( g_1`, g_2` ))=g_2 g_2` \rightarrow \phi(( g_1,g_2 )( g_1`, g_2` )\\
		Ker\phi=\{(g_1,g_2):\phi(g_1,g_2)=e  \}\rightarrow Ker\phi =\{ (g_1,e) \}=G_1 \\
		Im\phi=\{ \phi((g_1,g_2)) \}=\{g_2\}=G_2 \rightarrow G_1\times G_2/G_1\cong G_2$\\
		Следствия. Если $ G_1\lhd G, G/G_1\not\cong G_2 \rightarrow G \not\cong G_1\times G_2  $
		
		\color{red}{ACHTUNG} \color{black} $G/G_1 \cong G_2 \not\rightarrow G=G_1 \times G_2$ 
		
		Пример $C_4=\{e,a,a^2,a^3\} \qquad C_2=\{e,a^2\}\lhd C_4 \\
		C_4/C_2 =C_2 \qquad \text{,но } C_4 \not= C_2\times C_2$
		
		
		\hypertarget{th:keli}{\textbf{т. Кэлли}}
			$\forall$ конечная группа $G, |G|=n$ изоморфна подгруппе $S_n$
		
		
		$\lhd |G|=n \Rightarrow G=\{g_1,g_2,g_3, \dots , g_n\}  \\
		S(G)\text{-все биектифные функции на множестве G} \rightarrow S(G)=S_n \quad \phi : G\rightarrow S_n \qquad \phi_g=G \rightarrow G \quad  g\text{-фиксированный элемент}\in G \\ \phi_g(x)=g\lambda $
		\
		$\begin{Bmatrix}
		g_1 & g_2 & \dots & g_n \\
		\dots \\
		g_ig_1 & g_i g_2 & \dots & g_i g_n 
		\end{Bmatrix}$ 
		
		
		
		
		% % % % % % % % % %
		
		

		
		
		
	
		

		
	
			
			
		
		% % % % % % % % % %
		
		
		
		
	\newpage
		\section{Справочник.}
			\label{def:bin_oper}
			
	\newpage
		\section{Вопросы к коллоквиуму}
		\begin{enumerate}
			\item \hyperlink{indef:gruppoid}{Группоид}, \hyperlink{indef:halfgroup}{Полугруппа}, \hyperlink{indef:monoid}{Моноид}, \hyperlink{indef:group}{Группа}.
			\item \hyperlink{inpro:group}{Свойства групп}
			\item \hyperlink{indef:por_group}{Порядок группы(|G|)}, \hyperlink{indef:kpor_el}{Порядок элемента(|x|)}
			\item \hyperlink{indef:circle_group}{Циклическая группа}, \hyperlink{indef:inf_circle_group}{Конечность} и \hyperlink{indef:noninf_circle_gtoup}{Бесконечность} циклической группы.
			\item \hyperlink{f:circle}{Необходимые формулы и утверждения о циклической группе}
			\item \hyperlink{th:circle}{Теоремы и циклической группе}
			\item \hyperlink{indef_subgroup}{Определение подгруппы}, теоремы о равенствах \hyperlink{th:subgrop_e}{единичных элементов} и \hyperlink{th:subgroup_-1}{обратных элементов}, \hyperlink{th_subgroup_cri}{критерий подгруппы}
			\item \hyperlink{indef:equiv}{Отношение эквивалентности}
			\item \hyperlink{indef:left_class}{Левый} и \hyperlink{indef:right_class}{Правый} смежные классы и их \hyperlink{th:class-equiv}{Эквивалентность}
			\item \hyperlink{t_lang}{Теорема Лагранжа}
			\item \hyperlink{sl:t_lang}{Следствия из теоремы Лагранжа(в том числе Малая теорема Ферма, функция Эйлера, т. Вильсона)}
			\item \hyperlink{indef:norm_gr}{Нормальность группы}, \hyperlink{indef:fact_gr}{фактор-группа}
			\item \hyperlink{indef:morfizm}{Гомоморфизм, мономорфизм, эпиморфизм, эндоморфизм, автоморфизм, ядро и образ гоморфизма.}
			\item \hyperlink{inpro:gom}{Свойства гомоморфизма}
			\item \hyperlink{def:comp}{Внешнее прямое произведение групп}
			\item \hyperlink{inpro:out_comp}{Свойства внешнего произведения групп}
			\item \hyperlink{def:in_comp}{Внутреннее прямое произведение групп} 
			\item \hyperlink{inpro:in_comp}{Свойства внутреннего произведения групп}
			\item \hyperlink{th:keli}{т. Кэли}
			\item Изоморфизм внутренного и внешнего произведений
			\item \hyperlink{el}{Формула Эйлера,т. Эйлера, свойства ф. Эйлера,группа подстановок}
			\item \hyperlink{def:while}{Цикл, независимые циклы, транспозиция}
			\item \hyperlink{th:while}{Теоремы о циклах}
		\end{enumerate}
\end{document}