\documentclass[12pt]{article}
\usepackage[utf8x]{inputenc}
\usepackage[russian]{babel}
\usepackage{amsmath}
\usepackage{mathrsfs}
\usepackage{ dsfont }
\usepackage{amssymb}
%\usepackage{graphicx}
\usepackage{lipsum}
\usepackage{cases}
\usepackage{ wasysym }
\flushbottom 
\begin{document}
	\newtheorem{Th}{Теорема}
	\newtheorem{Def}{Определение}

Повторение.

\begin{enumerate}
	\item $n\mathds{Z}<\mathds{Z},+ ; a\sim b\leftrightarrow -a+b\in n\mathds{Z}\leftrightarrow b\in a+n\mathds{Z}$
	
	$T_a=\{b:a\sim b\}=a+n\mathds{Z}=\{a,a+n,a-n,a+2n,a-2n, \dots \}$
	
	$T_a=[a]_n=\bar{a}_n=[a]=\bar{a}=a$
	
	$[a]+[b]=[a+b]$
	
	$[a][b]={ab}$
	
	\begin{Th} 
	$	\left
			.\begin{aligned}
				a\sim a^\prime \\
				b\sim b^\prime
			\end{aligned}
		 \right \} $
		$\rightarrow$
		$\begin{cases}
		[a^\prime +b^\prime]=[a+b] \\
		[a^\prime b^\prime]=[ab]
		\end{cases}$
	\end{Th}
	$\LHD a^\prime =a+nl; b^\prime=b+nk \\
	a^\prime+b^\prime=a+b+n(l+k)\rightarrow [a^\prime+b^\prime]=[a+b] \\
	a^\prime b^\prime=ab+n(ak+bl+nlk)\rightarrow [a^\prime b^\prime]=[ab]
	\RHD$
	
	\item $H<G,\cdot$
	
	\begin{Th} 
		$|T_x|=|T_z|=|H|$
	\end{Th}
	
	$\LHD xh_1=xh_2 \rightarrow h_1=h_2 \RHD$
	
	\begin{Th}[Теорема Лагранжа] 
		Если $|G|=n<\infty \rightarrow |G|\vdots |H|$ 
	\end{Th}
	$\LHD \RHD$
	
	\textbf{Следствие из теоремы Лагранжа:} $|G|\vdots |x|$
	
	$\LHD x\rightarrow H=<x>, |H|=|x|
	\RHD$
	
\end{enumerate}

$\mathds{Z},n\mathds{Z}\rightarrow \frac{\mathds{Z}}{n\mathds{Z}}=\mathds{Z}_n=\{\bar{0},\bar{1},\bar{2},\bar{3}, \dots , \bar{n-1}\}$- кольцо остатсков при делении на $n$


\begin{Th}\label{thFirst}
	$\mathds{Z}_n$-поле $\leftrightarrow n$-простое 
\end{Th}

$\LHD \RHD$


\begin{Th}\label{thSecond}
	$\mathds{Z}_n \quad k-$обратим в $\mathds{Z}_n \leftrightarrow n$ и $k$-взаимно просты $\bigl\{ (n,k) =1 \bigr\} $
\end{Th}

\begin{Def}
	Функция Эйлера$\bigl\{\varphi (n) \bigr\} $ равна количеству натуральных чисел, меньших чем $n$ и взаимно простых с $n$.
\end{Def}

\begin{Def}
	$S_n$-группа подстановок(так же называют симметричной группой) \\
	$x=\{1,2,3, \dots , n\}$, $S_n$-мн-во биективных функций $\varphi :X\rightarrow X$ \\
	$\varphi= \begin{pmatrix}
	1 & 2 & 3 & 4 & \dots & n \\
	\varphi(1) & \varphi(2) & \varphi(3) & \varphi(4) & \dots & \varphi(n)
	\end{pmatrix}$
\end{Def}

\textbf{Примеры:} $\varphi=\begin{pmatrix}
1 & 2 & 3 & 4 \\
2 & 4 & 3 & 1
\end{pmatrix}$

$ \varphi(1)=2 , \varphi(2)=4, \varphi(3)=3, \varphi(4)=1$

$\phi=\begin{pmatrix}
1 & 2 & 3 & 4 \\
4 & 2 & 1 & 3
\end{pmatrix}$

$(\varphi\phi)(1)=\varphi(\phi(1))=\varphi(4)=1$

$\varphi^{-1}\varphi=\varphi\varphi^{-1}=e$

$e=\begin{pmatrix}
1 & 2 & 3 & 4 \\
1 & 2 & 3 & 4
\end{pmatrix}$

$\varphi^{-1}=\begin{pmatrix}
2 & 4 & 3 & 1 \\
1 & 2 & 3 & 3
\end{pmatrix}=\begin{pmatrix}
1 & 2 & 3 & 3 \\
4 & 1 & 3 & 2 
\end{pmatrix}$

$S_n$-группа  $\qquad |S_n|=n!$

\textbf{Цикл}

$\varphi=\begin{pmatrix}
1 & 2 & 3 & 4 \\
2 & 4 & 3 & 1
\end{pmatrix}=(124)(3)=(3)(124)$

$\phi=\begin{pmatrix}
1 & 2 & 3 & 4 \\
2 & 1 & 4 & 3
\end{pmatrix}=(12)(34)=(34)(12)$

\textbf{Независимые циклы}-называются циклы, если числа входят в один цикл, но не входят во второй цикл.

\begin{Th}
	Независимые циклы коммутируют друг с другом (или $\alpha ,\beta$-независимые циклы $\rightarrow \alpha\beta=\beta\alpha$ )
\end{Th}

\begin{Def}
	Циклом длины два называется транспозиция.
\end{Def}

\begin{Th}
	Если $\alpha=(i_1,i_2, \dots ,i_k)$-цикл длины $k \rightarrow |\alpha|=k$
\end{Th}

\begin{Th}
	Пусть $\varphi=\alpha_1\alpha_2\dots\alpha_n$- произведение независимых циклов.\\
	$\alpha_1=(i_1,i_2,\dots, i_k)$\\
	$\alpha_2=(j_1,j_2,\dots, j_l) \rightarrow |\varphi|=$НОК$(|\alpha_1|,|\alpha_2|,\dots,|\alpha_m|)$
\end{Th}


\end{document}