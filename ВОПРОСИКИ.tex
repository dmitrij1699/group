\documentclass[12pt]{article}
\usepackage[utf8]{inputenc}
\usepackage[russian]{babel}
\usepackage{amsmath}
\usepackage{dsfont}
\usepackage{amssymb}
\usepackage{hyperref}
\usepackage{lipsum}
\usepackage{cases}
\usepackage{ wasysym }
\usepackage[usenames]{color}
\usepackage{xcolor}
\usepackage{colortbl}
\usepackage{mathrsfs}
%\righthyphenmin=2
%\definecolor{linkcolor}{HTML}{799B03}
%\definecolor{urlcolor}{HTML}{799B03}

\begin{document}
						
		\begin{enumerate}
				\item Группоид
				\item Полугруппа
				\item Моноид
				\item Группа
				\item Порядок группы(|G|)
				\item Порядок элемента(|x|)
				\item Циклическая группа
				\item Определение подгруппы
				\item Чему равен $|x^k|$? 
				\item Критерий подгруппы
				\item Отношение эквивалентности
				\item Левый и Правый смежные классы 
				\item Теорема Лагранжа
				\item Малая теорема Ферма
				\item Функция Эйлера
				\item т. Эйлера 
				\item т. Вильсона
				\item Нормальность группы 
				\item фактор-группа
				\item Автоморфизм
				\item AutG 
				\item Внутренний автоморфизм
				\item IntG
				\item Гомоморфизм
				\item Мономорфизм
				\item эпиморфизм
				\item эндоморфизм 
				\item ядро гомоморфизма 
				\item образ гомоморфизма
				\item Внешнее прямое произведение групп
				\item Внутреннее прямое произведение групп
				\item т. Кэли				
				\item группа подстановок
				\item Цикл 
				\item Независимые циклы
				\item Транспозиция
				\item Определение полупрямого произведения
				\item Четность и нечетность подстановки
				\item Знак подстановки
				\item Знак произведения подстановок 
				\item Длина цикла
				\item Порядок цикла
				\item Порядок произведения независимых циклов
			\end{enumerate}
	
\end{document}