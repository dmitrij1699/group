\documentclass[12pt]{article}
\usepackage[utf8x]{inputenc}
\usepackage[russian]{babel}
\usepackage{amsmath}
\usepackage{mathrsfs}
\usepackage{ dsfont }
\usepackage{amssymb}
\usepackage{graphicx}
\usepackage{lipsum}
\flushbottom 
\begin{document}
	Основные источники: \begin{enumerate}
	\item http://elisey-ka.ru/matan/index.htm
	\item Высшая математика Шипачев В.С. 2005 -479с 
	\item http://us.chem.msu.su/Lection/Math1/index.htm
	\end{enumerate}
	\section{Билет. Понятие предела последовательности. Основные теоремы о пределах}
	
	Число a называется пределом последовательности ${x_n}$, если для любого положительного числа $\epsilon$ существует номер N такой, что при n>N выполняется неравенство $|x_n-a|<\epsilon$ 
	
	Основные теоремы о пределах: 
	
	
	\begin{enumerate}
	    \item Предел постоянной величины равен ей самой.
	%	\item Предел суммы двух функций равен сумме их пределов, если
	%	эти пределы конечны.
	%	\item Предел произведения двух функций равен произведению их
	%	пределов, если эти пределы конечны.
	%	\item Предел отношения двух функций равен отношению их пре-
	%	делов, если эти пределы конечны и предел делителя не равен нулю. 
		\item Сходящаяся последовательность имеет только один предел
		\item Сходящаяся последовательность ограничена
		\item Сумма(разность) двух сходящихся последовательностей ${x_n}$ и ${y_n}$ есть сходящаяся последовательность, предел которой равен сумме(разности) пределов последовательностей ${x_n}$ и ${y_n}$.
		\item Произведение сходящихся последовательностей ${x_n}$ и ${y_n}$ есть сходящаяся последовательность, предел которой равен произведению пределов послдеовательностей ${x_n}$ и ${y_n}$.
		\item Частное двух сходящихся последовательностей ${x_n}$ и ${y_n}$ при условии, что предел ${y_n}$ отличен от нуля, есть сходящаяся последовательность, предел которой равен частному пределов последовательностей ${x_n}$ и ${y_n}$.
		
	\end{enumerate}

	\section{Билет. Сущестоввание предела в монотонной ограниченной последовательности. Число e.} 
	
	
	
\begin{figure}

\includegraphics[width=0.7\textwidth]{teorema}
\end{figure}
\begin{figure}
	
	\includegraphics[width=0.7\textwidth]{teorema1}
\end{figure}

\begin{figure*}
\includegraphics[width=0.7\textwidth]{teorema2}
\end{figure*}


\newpage

 
 
	\section{Билет. Лемма о вложенных отрезках.Теорема Больцано-Вейерштрасса}
	
	\textbf{Теорема о вложенных отрезках:}Для любой последовательности вложенных отрезков существует единственная точка, принадлежащая всем отрезкам этой последовательности.\\
	\textbf{Теорема Больцано-Вейерштрасса:}Из всякой ограниченной последовательности точек пространства $R^n$  можно выделить сходящуюся подпоследовательность.
	
	\section{Билет. Последовательность Коши(Фундаментальная последовательность).Критерий Коши.}
	
	\textbf{Фундаментальная последовательность} (последовательность Коши, сходящаяся в себе последовательность)  – последовательность  {xn}, удовлетворяющая следующему условию Коши:
	Для любого $\epsilon > 0$ существует такое n, что для всех $n > N, m > N$ выполняется неравенство $|x_n - x_m| < \epsilon$. \\
	\textbf{Коши критерий сходимости последовательности} Пусть задана числовая последовательность ${x_n}$. Эта последовательность сходится тогда и только тогда, когда для любого числа $\epsilon > 0$ существует номер N такой, что при всех $n > N$  и любых натуральных m выполняется неравенство $|x_{n+m} -x_n|<\epsilon$      (т.е. расстояние между членами последовательности с номерами n  и $n+m$ меньше $\epsilon$)
	
	\section{Билет. Предел функции. Эквивалентность определений. Предел функции через сходящиеся последовательности. Предел по Гейне/Коши.}
	
	Число A называется пределом функции f(x) в точке $x=x_0$, если для любой сходящейся к $x_0$ последовательности значений аргумента x, отличных от $x_0$, соответствующая последовательность значений функции сходится к числу A.(Гейне)
	
	Число A называется пределом функции f(x) в точке $x=x_0$,если для любого числа $\epsilon>0$ существует число $\sigma>0$ такое, что для всех $ x \in X , x \not= x_0$, удовлетворяющих неравенству $|x-x_0|<\sigma$, выполняется неравенство $|f(x)-A|<\epsilon$.(Коши)
	
	\section{Билет. Основные свойства предела функции.}
	
	\begin{enumerate}
		\item Предел суммы равен сумме пределов, если каждый из них существует
		\item Предел разности равен разности пределов, если каждый из них существует
		\item Предел постоянной величины равен самой постоянной величине
		\item Предел произведения функции на постоянную величину.Постоянный коэффициэнт можно выносить за знак предела
		\item Предел произведения равен произведению пределов, если каждый из них существует
		\item Предел частного равен частному пределов, если каждый из них существует и знаменатель не обращается в нуль
		\item Первый и второй зам. пределы.
		
		
		
		
	\end{enumerate}
	
	
	
	\section{Билет. Критерий Коши о существовании предела функции.}
	\begin{figure}
	\centering
	\includegraphics[width=0.8\linewidth]{27}
	\caption[7 билет]{7 билет.Критерий Коши}
	\end{figure}
	Изображение снизу.
	\newpage
	\section{Непрерывные функции.Свойства непрерывных функций на отрезке.}
	
	Функция f(x) называется непрерывной в точке $x_0$, если предел функции и её значение в этой точке равны.\\
	Функция f(x) называется непрерывной в точке $x_0$, если для любого $\epsilon >0$ существует $\sigma >0$ такое, что для всех x, удовлетворяющих неравенству $|x-x_0|<\sigma$, выполняется неравенство $|f(x)-f(x_0)|<\epsilon$.
	
	\begin{enumerate}
		\item \textbf{Первая теорема Вейерштрасса.}Если функция непрерывна на отрезке, то она достигает на этом отрезке свои наибольшее и наименьшее значения.
		\item Непрерывная на отрезке $[a;b]$ функция является ограниченной на этом отрезке.
		\item \textbf{Вторая теорема Больцано-Коши.}  Если функция $y=f(x)$  является непрерывной на отрезке $[a;b]$  и принимает на концах этого отрезка неравные между собой значения, то есть $f(a)=a_0 ,f(b)=b_0$  , то на этом отрезке функция принимает и все промежуточные значения между $a_0$  и $b_0$  .
		\item \textbf{Первая теорема Больцано-Коши.}Если функция $y=f(x)$, которая непрерывна на некотором отрезке $[a;b]$ , принимает на концах отрезка значения разных знаков, то существует такая точка $c \in [a;b]$  такая, что $f(c)=0$  .
	\end{enumerate}
	
	
	\section{Билет. Равномерная непрерывность. Теорема Кантора(Кантора-Гейне?).}
	
\begin{figure}
\centering
\includegraphics[width=0.7\linewidth]{38}
\end{figure}

\newpage

	\section{Билет. Открытое и замкнутое множество. Граница и замыкание множества. Композиция множеств.}	
	
	Множество называется замкнутым, если оно содержит все свои предельные точки. Если множество не имеет ни одной предельной точки, то его тоже принято считать замкнутым. Кроме своих предельных точек, замкнутое множество может также содержать изолированные точки. Множество называется открытым, если каждая его точка является для него внутренней.
	
	?????
	
	\section{Билет. Точки разрыва и их классификация.}
	
	Точка $x_0$ называется точкой разрыва функции $f(x)$, если $f(x)$ в точке $x_0$ не является непрерывной.
	
	\textbf{Разрыв 1-го рода.}Точка $x_0$ называется точкой разрыва 1-го рода функции $f(x)$ ,если в этой точке функция $f(x)$ имеет конечные, но не равные друг другу правый и левый пределы.
	
	\textbf{Разрыв 2-го рода.}Точка $x_0$ называется точкой разрыва 2-го рода функции $f(x)$ ,если в этой точке функция $f(x)$ не имеет по крайней мере одного из односторонних пределов или хотя бы один из односторонних пределов бесконечен.
	
	\section{Билет. Производная, её геометрический и механический смысл.}
	
	Производной функции $y=f(x)$ в точке $x_0$ называется предел при $\Delta x \Rightarrow0$ отношения приращения функции в этой точке к приращению аргумента(при условии, что этот предел существует).
	
	геометрический-касательная, физический- скорость, путь и прочее.
	
	\section{Билет. Теорема о связи непрерывности и дифференцируемости. Райский билет :)}
	
	Если функция   дифференцируема в некоторой точке  , то она непрерывна в этой точке. 
	
	\section{Билет. Арифметические действия с производными.}
	\begin{enumerate}
		\item Производная от константы равна нулю.
		\item Константу можно вынести за знак производной.
		\item Производная суммы любого числа функций равна сумме производных этих функций.
		\item Производная произведения двух функций равна \newline $(f(x)g(x))'=f'(x)g(x)+f(x)g'(x)$
		\item Производная частного двух функций равна $(\frac{f(x)}{g(x)})'=\frac{f'(x)g(x)-f(x)g'(x)}{g^2(x)} $
		\item Пусть $y=f(u)$, где $u=j(x)$, тогда $y'=f'(u)j'(x)$
		
		
	\end{enumerate}
	
	\section{Билет. Таблица производных.}

	\includegraphics[width=0.5\linewidth]{tab_proizv}
	

	\section{Билет. Производные сложной и обратной функции.}

	
	Производная сложной функции: $(u(v))'=u'(v)*v'$
	
	
	Пусть функция $x=f(y)$ монотонна и дифференцируема в некотором интервале $(a,b)$ и имеет в точке у этого интервала производную $f'(y)$, не равную нулю. Тогда в соответствующей точке $x$ обратная функция $y=f^{-1}(x)$ имеет производную $[f^{-1}(x)]'$, причем $[f^{-1}(x)]'=\frac{1}{f'(y)}$ или $y_x'=\frac{1}{x_y'}$
	
	
	\section{Билет. Дифференциал, его связь с производной, геометрический смысл, инвариантность.}
	
	Дифференциал  — линейная часть приращения функции. $dy=f'(x)\Delta x$
	
	Дифференциал функции $y = f(x)$ равен приращению ординаты касательной , проведенной к графику этой функции при изменении аргумента на $\Delta x$.
	
	Пусть существует сложная функция $z=f(g(x))$ , и существует ее производная:$z_x'=z_y'y_x'$ . Считая $y$ независимой переменной, получим формулу дифференциала:$dz=z_y'dy$ . Теперь, если считать $y$ зависимой от $x$, получим:$dz=z_y'y_x'dx=z_y'dy$ , т.к.$dy=y_x'dx$ . То есть получается, что формула дифференциала не зависит от типа переменной.
	
	Не взирая на то, является ли переменная x зависимой или нет, для вычисления дифференциала используется единая формула - инвариантность формул.
	
	\section{Билет. Теорема Ролля, её геометрический смысл.}
	
	\textbf{Теорема Ролля.} (О нуле производной функции, принимающей на концах отрезка равные значения)
	
	Пусть функция $y=f(x)$ \begin{enumerate}
	\item непрерывна на отрезке $[a;b]$ ;
	\item дифференцируема на интервале $(a;b)$;
	\item на концах отрезка $[a;b]$ принимает равные значения $f(a)=f(b)$ . \end{enumerate}
	Тогда на интервале $(a;b)$ найдется, по крайней мере, одна точка $x_0$ , в которой $f'(x_0)=0$.
	
	\textbf{Геометрический смысл теоремы Ролля:}Найдется хотя бы одна точка, в которой касательная к графику функции будет параллельна оси абсцисс.
	
	
	
	\section{Билет. Теорема Лагранжа, её геометрический смысл. Теорема Коши.}
	
	\textbf{Теорема Лагранжа:}Пусть функция $y=f(x)$   \begin{enumerate}
	\item непрерывна на отрезке $[a;b]$ ;
	\item дифференцируема на интервале $(a;b)$.  \end{enumerate}
	Тогда на интервале $(a;b)$ найдется по крайней мере одна точка $x_0$ , такая, что $\frac{f(b)-f(a)}{b-a}=f'(x_0)$
	
	\includegraphics[width=0.9\linewidth]{teorema3}
	
	\textbf{Теорема Коши:}Если функции $y=f(x)$ и $y=g(x)$ :  \begin{enumerate}
	\item непрерывны на отрезке $[a;b]$;
	\item дифференцируемы на интервале $(a;b)$;
	\item производная $g'(x) \not= 0$ на интервале $(a;b)$ , \end{enumerate}
	тогда на этом интервале найдется по крайней мере одна точка $x_0$  , такая, что
	$\frac{f(b)-f(a)}{g(b)-g(a)}=\frac{f'(x_0)}{g'(x_0)}$
	
	\section{Билет. Правило Лопиталя.}
	
	\textbf{Теорема Лопиталя}. Если: \begin{enumerate}
		\item $\lim_{x\to a}f(x)=\lim_{x\to a}g(x)=0$ или $\infty$
		\item $f(x)$ и $g(x)$ дифференцируемы в окрестности $a$
		\item $g'(x)\not= 0$ в окрестности $a$
		\item существует $\lim_{x\to a}\frac{f'(x)}{g'(x)}$
	\end{enumerate} то существует $\lim_{x\to a}\frac{f(x)}{g(x)}=\lim_{x\to a}\frac{f'(x)}{g'(x)}$
	
	\section{Билет. Многочлен Тейлора, формула Тейлора}
	
	Общий вид формулы Тейлора:$f(x)=T_n(x)+R_n(x)$ , где \newline $T_n(x)=\sum_{k=0}^{n}\frac{f^k(x_0)}{k!}*(x-x_0)^k$ - многочлен Тейлора. Для того, чтобы написать многочлен Тейлора степени $n$, необходимо наличие $n$ производных в точке $x_0$.$R_n(x)$  - остаточный член Тейлора. Остаточный член имеет различный вид в зависимости от требований. 
	
	\section{Билет. Остаточный член формулы Тейлора в формах Пеано и Лагранжа.}
	
	\textbf{Форма Лагранжа:}$R_n(x)=\frac{f^{(n+1)}(c)}{(n+1)!}*(x-x_0)^{n+1}$
	
	\textbf{Форма пеано:}$R_n(x)=? ((x-x_0)^n), x\to x_0$ 
	
	\section{Билет. Локальный экстремум функции одного переменного. Необходимое и достаточное условия экстремума.}
	
	Пусть функция $y=f(x)$ определена в некоторой $\delta$-окрестности точки $x_0$, где $\delta>0$. Говорят, что функция $f(x)$ имеет локальный максимум в точке $x_0$, если для всех точек $x\not=x_0$, принадлежащих окрестности $(x_0−\delta,x_0+\delta)$, выполняется неравенство $f(x)\le f(x_0)$. Если для всех точек $x\not=x_0$ из некоторой окрестности точки $x_0$ является точкой строгое неравенство $f(x)<f(x_0)$,то точка $x_0$ является точкой локального максимума. \\
	Аналогичено определяется локальный минимум функции $f(x)$.В этом случае для всех точек $x\not=x_0$ из $\delta$-окрестности $(x_0-\delta,x_0+\delta)$ точки $x_0$ справедливо неравенство $f(x)\ge f(x_0)$.Соответственно строгий локальный минимум описывается строги неравенством $f(x)>f(x_0)$
	
	\textbf{Необходимое условие:}Если точка $x_0$ является точкой экстремума функции $f(x)$, то в этой точке либо производная равна нулю, либо не существует. Другими словами, экстремумы функции содержатся среди ее критических точек. 
	
	\textbf{Достаточные условия:} \begin{enumerate}
		\item Пусть функция $f(x)$ дифференцируема в некоторой окрестности точки $x_0$, кроме, быть может, самой точки $x_0$, в которой, однако, функция непрерывна. Тогда:\\
		Если производная $f′(x)$ меняет знак с минуса на плюс при переходе через точку $x_0$, (слева направо), то точка $x_0$ является точкой строгого минимума. Другими словами, в этом случае существует число $\delta>0$, такое, что
		$\forall x \in (x_0-\delta,x_0) \Rightarrow f'(x)<0$,
		$\forall x \in (x_0,x_0+\delta) \Rightarrow f'(x)>0$.
		
		Если производная $f′(x)$, наоборот, меняет знак с плюса на минус при переходе через точку $x_0$, то точка $x_0$ является точкой строгого максимума. Иначе говоря, существует число $\delta>0$, такое, что
		$\forall x \in (x_0-\delta,x_0) \Rightarrow f'(x)>0$,
		$\forall x \in (x_0,x_0+\delta) \Rightarrow f'(x)<0$.
		\item Пусть в точке $x_0$ первая производная равна нулю: $f(x_0)=0$, т.е. точка $x_0$ является стационарной точкой функции $f(x)$. Пусть также в этой точке существует вторая производная $f''(x_0)$. Тогда:
		Если $f''(x_0)>0$, то $x_0$ является точкой строгого минимума функции $f(x)$;
		Если $f''(x0)<0$, то $x_0$ является точкой строгого максимума функции $f(x)$.
		\item Пусть функция $f(x)$ имеет в точке $x_0$ производные до $n$-го порядка включительно. Тогда, если
		$f'(x_0)=f''(x_0)=\dots=f^{(n−1)}(x_0)=0иf^(n)(x_0)\not=0$,
		то при четном $n$ точка $x_0$ является\\
		точкой строгого минимума, если $f^(n)(x_0)>0$, и\\
		точкой строгого максимума, если $f^(n)(x_0)>0$.\\
		При нечетном $n$ экстремума в точке x0 не существует. 
		
		Ясно, что при $n=2$ в качестве частного случая мы получаем рассмотренное выше второе достаточное условие экстремума. Чтобы исключить такой переход, в третьем признаке полагают, что $n>2$. 
	\end{enumerate}
	
	\section{Билет. Геометрический смысл второй производной. Точки перегиба.}
	
	
	
	
	
	\section{Билет. Ассиптоты графика функции. Существование наклонной ассимптоты.}
	
	
	
	
	
	\section{Билет. Частные производные функции нескольких переменных. Дифференциал.}
	
	
	
	
	\section{Билет. Формула Тейлора для функции нескольких переменных.}
	
	
	

	
	\section{Билет. Локальный экстремум функции нескольких переменных. Необходимое условие экстремума.}
	
	
	
	
	
	
	
	
	
	
	
	
	
	
	
	
	
	
	
	
	
\end{document}